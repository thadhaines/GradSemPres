\documentclass[14pt, unknownkeysallowed]{beamer}

\usepackage{tikz}
\usepackage{comment}
\usepackage{fancyvrb}
\usepackage{tikz}
\usetikzlibrary{arrows,decorations.pathmorphing,backgrounds,positioning,fit,petri}
\usepackage{circuitikz} % for circuits!
\usetikzlibrary{arrows.meta} % for loads
\usepackage{float}
\usepackage{listings}

\usepackage{hyperref}

\usepackage[default]{berasans}
\renewcommand*\familydefault{\sfdefault}  %% Only if the base font of the document is to be sans serif
\usepackage[T1]{fontenc}

\setbeamertemplate{navigation symbols}{}
\usetheme{Dresden}
\usecolortheme{seagull}
\usefonttheme{professionalfonts}


%\title{Long-Term Simulation of Power System Dynamics using Time Sequenced Power Flows}
\title{Long-Term Power System Dynamic Simulation using Time Sequenced Power-Flows}
\author{Thad Haines}
\institute[MT TECH]{Montana Technological University - Master's Thesis Research Project}
\date{October 22nd, 2019}

\newcounter{assumptions}

\begin{document}
	
\begin{frame}
	\titlepage
\end{frame}

%************************************************
\section{Introduction}
%________________________________________________
\subsection{Explanation of Project}
%------------------------------------------------
\begin{frame}
What is a Power System?
\end{frame}
%------------------------------------------------
\begin{frame}
What is Dynamic Simulation?
\end{frame}
%------------------------------------------------
\begin{frame}
What is Long-Term?
\end{frame}
%------------------------------------------------
\begin{frame}
What is a Power Flow?
\end{frame}
%------------------------------------------------
\begin{frame}
What are Time Sequenced Power Flows?
\end{frame}
%------------------------------------------------
\begin{frame}
Putting it all together...
\end{frame}
%------------------------------------------------
\begin{frame}
Project Goals:
\begin{itemize}
\item Develop software for (long-term dynamic) LTD simulations.
\item Incorporate useful parts of GE software (PSLF):
\begin{itemize}
	\item Power Systems ($.sav$ files)
	\item Dynamic model data ($.dyd$ files)
	\item Power-flow solver
\end{itemize}
\item Create simplified dynamic models compatible with LTD time steps.
\item Investigate long-term events.
\begin{itemize}
	\item Turbine governor and AGC interaction
	\item System response to wind ramps
\end{itemize}
\end{itemize}
\end{frame}
%------------------------------------------------

%************************************************
\section{Simulation Model}
%________________________________________________
\subsection{Assumptions, Coding Decisions, Approaches, and Software Operation.}
%------------------------------------------------
\begin{frame}
This simulation assumes:
\begin{enumerate}
\item Time steps of 0.5 to 1 second.
\item Fast dynamics are 'mostly' ignored.
\item System remains synchronized.
\item System frequency is described by the combined PU swing equation:
\setcounter{assumptions}{\value{enumi}} % allow for break in counting
\end{enumerate}
\[ \dot{\omega}_{sys} = \dfrac{1}{2H_{sys} } \left( \dfrac{P_{acc, sys} }{\omega_{sys}(t)} - D_{sys}\Delta\omega_{sys}(t)  \right)\] 
\begin{enumerate}
\setcounter{enumi}{\value{assumptions}}
\item No system damping $(D_{sys} = 0)$.
\end{enumerate}
\end{frame}
%------------------------------------------------

\begin{frame}
Software used:
\begin{itemize}
\item Python 3, IronPython
\item Erlang / AMQP
\item MATLAB
\end{itemize}
\end{frame}
%------------------------------------------------
\begin{frame}
\begin{itemize}
\item Agent\\ An autonomous individual object with properties and methods in a computer simulation.
\item Agent-Based Modeling\\ The idea that a system can be modeled using agents in an environment, and a description of agent-agent and agent-environment interactions. \tiny[2]
\end{itemize}
\end{frame}
%------------------------------------------------

%************************************************
\section{Current Conclusions}
%------------------------------------------------
\begin{frame}
Current Conclusions
\begin{itemize}
	\item Software (PSLTDSim) produces valid output for small to medium size systems.
	\item Governor and AGC interactions can happen easily
	\item Deadbands and conditional logic should be used to limit governor and AGC conflicts
\end{itemize}
\end{frame}
%------------------------------------------------
\begin{frame}
Future Work
\begin{itemize}
\item Incorporation of more PSLF type objects into simulation.
\item Creation of additional dynamic models and agents.
\item Addition of definite time controller.
\item Departure from reliance on GE software.
\end{itemize}
\end{frame}
%------------------------------------------------
\begin{frame}
References\vspace{1em}\\
%\resizebox{.8\textwidth}{.4\textheight}{
\begin{minipage}{\textwidth}
\footnotesize
\begin{itemize}
\item[[1] GE Energy. "Mechanics of Running PSLF Dynamics" Phoenix, AZ, 2015
\item[[2] Rand, W. (2018). Agent-Based Modeling: What is Agent-Based Modeling? [Online] Available: https://www.youtube.com/watch?v=FVmQbfsOkGc
\item[[3] P.M. Anderson and A.A. Fouad, Power System Control and Stability, 2nd ed. IEEE Press, 2003, p20.
\end{itemize}
\end{minipage}
%}
\end{frame}
\end{document}