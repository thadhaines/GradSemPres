\documentclass[14pt, unknownkeysallowed]{beamer}

\usepackage{tikz}
\usepackage{comment}
\usepackage{fancyvrb}
\usepackage{tikz}
\usetikzlibrary{arrows,decorations.pathmorphing,backgrounds,positioning,fit,petri}
\usepackage{circuitikz} % for circuits!
\usetikzlibrary{arrows.meta} % for loads
\usepackage{float}
\usepackage{listings}

\usepackage{hyperref}
\usepackage{multicol}

% For figure usage and linking
\usepackage{graphicx}
\graphicspath{ {figures/} }

% bibiliography
\usepackage[backend=biber,style=ieee, % bibliography style may be changed
sorting=nyt % sorts by author, year, title
]{biblatex}
\addbibresource{../Thesis/template/biliography.bib}
\nocite{*} % include non referenced citations
\setbeamertemplate{bibliography item}{\insertbiblabel} % number refs...?

\usepackage[default]{berasans}
\renewcommand*\familydefault{\sfdefault}  %% Only if the base font of the document is to be sans serif
\usepackage[T1]{fontenc}

\setbeamertemplate{navigation symbols}{}
\usetheme{Dresden}
\usecolortheme{seagull}
\usefonttheme{professionalfonts}


%\title{Long-Term Simulation of Power System Dynamics using Time Sequenced Power Flows}
\title{Power System Long-Term Dynamic Simulation using Time-Sequenced \\Power Flows}
\author{Thad Haines}
\institute[MT TECH]{Montana Technological University - Master's Thesis Research Project}
\date{October 22nd, 2019}

\newcounter{assumptions}

\begin{document}
	
\begin{frame}
	\titlepage
\end{frame}

%************************************************
\section{Power System}
%________________________________________________
\subsection{Physical Structure}
%------------------------------------------------
\begin{frame}
\frametitle{What is a Power System?}
\includegraphics[width=\linewidth]{largeGrid} % image  fromhttps://www.ferc.gov/industries/electric/indus-act/reliability/blackout/ch1-3.pdf
% 3 Area MiniWECC picture?
Electrical supply connected to demand.\\
%\vspace{1em}
%Power plants connected via transmission lines to cities in different areas. \\
%\vspace{1em}
%A part of `The Grid'.
% picture of north american interconnections
\end{frame}
%------------------------------------------------
\begin{frame}
\frametitle{U.S. Electric Generation}
{\centering
\includegraphics[height=.9\textheight]{july2019map} % from https://www.eia.gov/electricity/data/eia860m/
}
\end{frame}
%------------------------------------------------
\begin{frame}
\frametitle{U.S. Electric Transmission Lines}
{\centering
\includegraphics[width=\linewidth]{UStransmission} %from https://www.eia.gov/state/maps.php?v=Electricity
}
\end{frame}
%------------------------------------------------
\begin{frame}
\frametitle{Electric Transmission Lines}
{\centering
\includegraphics[height=.9\textheight]{xgridPlusSubs} % from Physical Security of the U.S. Power Grid.pdf

}
\end{frame}
%------------------------------------------------
\begin{frame}
\frametitle{Interchanges}
Interchanges are physical things - it makes sense to bring up here
%{\centering
%\includegraphics[height=.9\textheight]{xgridPlusSubs} % from Physical Security of the U.S. Power Grid.pdf
%
%}
\end{frame}
%------------------------------------------------
\begin{frame}
\frametitle{WECC}
Current physical location, and it's what I'm simulating, good idea to bring up here.
%{\centering
%\includegraphics[height=.9\textheight]{xgridPlusSubs} % from Physical Security of the U.S. Power Grid.pdf
%}
\end{frame}
%------------------------------------------------
%________________________________________________
\subsection{Operational Structure}
%------------------------------------------------
\begin{frame}
\frametitle{`People in Charge'}
\begin{itemize}
\item \textbf{FERC}{ \footnotesize Federal Energy Regulatory Commission}\\
 Part of the Department of Energy
\item \textbf{NERC}{ \footnotesize
 North American Electric Reliability Corp.}\\
 Aurthority granted by FERC
 \item \textbf{Balancing Authorities} \\
 Manage specific portions of the power system to balance supply and demand and maintain mandatory operating conditions set by FERC and NERC. % abstracted from: https://www.eia.gov/todayinenergy/detail.php?id=27152
\end{itemize}
\end{frame}
%------------------------------------------------
\begin{frame}
\frametitle{Six NERC Regions} % how much does this matter?
{\centering
\includegraphics[height=.8\textheight]{NERCregions} % from https://www.nerc.com/AboutNERC/keyplayers/Pages/default.aspx
}
\end{frame}
%------------------------------------------------
\begin{frame}
\frametitle{Main Interconnections} % this may be more of a rehash
{\centering
\includegraphics[height=.85\textheight]{NERCInterconnectionsEDIT} % edited from https://www.nerc.com/AboutNERC/keyplayers/Pages/default.aspx
}
\end{frame}
%------------------------------------------------
\begin{frame}
\frametitle{Balancing Authorities (BAs)}\ \vspace{.5em}
{\centering
\href{https://www.eia.gov/realtime_grid/}%
{\includegraphics[width=.95\linewidth]{BAs}} % from https://www.eia.gov/realtime_grid/
}
\end{frame}
%------------------------------------------------
\begin{frame}
\frametitle{BA Action - Forcasting} \ \vspace{.5em}
{\centering
{\includegraphics[height=.4\textheight]{chart} 
\includegraphics[height=.4\textheight]{chart1}} % https://www.eia.gov/realtime_grid/#/data/graphs?end=20190704T18&start=20190627T18&bas=000001&regions=0&errSeriesType=S
}
\end{frame}
%------------------------------------------------
\begin{frame}
\frametitle{BA Action - Import \& Export} \ \vspace{.5em}
{\centering
{\includegraphics[height=.4\textheight]{chart2} 
\includegraphics[height=.4\textheight]{chart3}} % https://www.eia.gov/realtime_grid/#/data/graphs?end=20190704T18&start=20190627T18&bas=000001&regions=0&errSeriesType=S
}
\end{frame}
%------------------------------------------------
\begin{frame}
\frametitle{BA Action - Interchange Error} \ \vspace{.5em}
{\centering
{\includegraphics[height=.6\textheight]{chart4}} % https://www.eia.gov/realtime_grid/#/data/graphs?end=20190704T18&start=20190627T18&bas=000001&regions=0&errSeriesType=S
}
\end{frame}
%------------------------------------------------

%************************************************
\section{Long-Term Dynamics}
%________________________________________________
\subsection{Explanation of Wording}
%------------------------------------------------
\begin{frame}
\frametitle{What is Dynamic Simulation?}
A computer's mathematical estimation of how a system will change over time.\\% in response to certain inputs and known starting conditions.\\ % Simulation of system changing over time
\vspace{1em}
Think solving ODE's.\\
\vspace{1em}
How certain qualities of a power system may change over time in response to a known perturbance.
\end{frame}
%------------------------------------------------
\begin{frame}
\frametitle{What is Long-Term?}\ \vspace{-1em}
% Insert timeScales Graph Here
\begin{minipage}{.7\linewidth}
{\centering
{\includegraphics[width=.8\linewidth]{timeScales}} % from p3. chow sauer pai , Power System Dynamics and Stability book
}{\footnotesize\cite{SauerPaiChow}} 
\end{minipage}% 
%Long-term describes the amount of time required for events of interest to occur and the simulation length to be executed. \\
%In this case: 10 to 60 minutes
\begin{minipage}{.4\linewidth}
\hspace{-1em}
\begin{itemize}
\item 1 sec time step
\item 10-60 minute simulations
\end{itemize}
\end{minipage}


\end{frame}

%------------------------------------------------
\subsection{Key Dynamic Concepts of Interest}
%------------------------------------------------
\begin{frame}
\frametitle{Generators}
Frequency, Accelerating power and Inertia.\\
\[ \dot{\omega}_{sys} = \dfrac{1}{2H_{sys} } \left( \dfrac{P_{acc, sys} }{\omega_{sys}(t)} - D_{sys}\Delta\omega_{sys}(t)  \right)\]
Direct link - electric demand always met. If there isn't enough generation, the kinetic energy stored as a moving inertia in a generator is converted to electric energy and the generator slows down.
\end{frame}
%------------------------------------------------
\begin{frame}
\frametitle{Control Reaction Times}
primary, secondary, tertiary

picture...
\end{frame}
%------------------------------------------------
\begin{frame}
\frametitle{Turbine Speed Governors}
(Governors)
Turbine speed governors adjust a machines mechanical power to stop frequency decline. Input is frequency deviation and current operating set point. Classified Primary control.
% model of tgov1?
{\centering\includegraphics[width=\linewidth]{../../ResearchDocs/TEX/models/tgov1/tgov1}}
\end{frame}
%------------------------------------------------
\begin{frame}
\frametitle{Automatic Generation Control}
%(AGC), also knows as Load Frequency Control or LFC)
%\\
%Adjusts generator nominal operating set point to remove any inadvertent interchange and restore system frequency to 60 Hz. Classified Secondary Control.
% kundur pic... p378 firug 9.1
{\centering\includegraphics[height=.8\textheight]{AGCblockdiagram}}
{\footnotesize\cite{Kundur}}
\end{frame}
%------------------------------------------------
\begin{frame}
\frametitle{Automatic Generation Control}
%(AGC), also knows as Load Frequency Control or LFC)
%\\
Adjusts generator nominal operating set point to remove any inadvertent interchange and restore system frequency to 60 Hz. Classified Secondary Control.

ACE Conventions Positive ACE denotes over generation. $B$ (the frequency bias) is negative.
\begin{align*}
\text{ACE}_{\text{tie line}} &= P_{\text{interchange}} - P_{\text{sched interchange}}\\
\text{ACE}_{\text{frequency bias}} &= 10B(f_{\text{actual}}-f_{\text{sched}})f_{base}\\
\text{ACE} &= \text{ACE}_{\text{tie line}} -\text{ACE}_{\text{frequency bias}}
\end{align*}

\end{frame}
%------------------------------------------------
\begin{frame}
\frametitle{Multi-Area Interactions}

Areas import or export power to each other.
% reuse one of those chart pictures
\end{frame}
%************************************************
\section{Time-Sequenced Power Flows}
%________________________________________________
\subsection{Explanation of Computational Approach}
%------------------------------------------------
\begin{frame}
\frametitle{What is a Power Flow?}
A steady state solution to all bus voltages, bus voltage angles, and real  and reactive power of a system.\\
\vspace{1em}
A \emph{snapshot} of a power system. \\% not dynamic
\vspace{1em}
Power flows are do not care about time.
\end{frame}
%------------------------------------------------
\begin{frame}
\frametitle{Time-Sequenced Power Flows?}
Multiple power flows arranged in a way to give the allusion of time.\\
\vspace{1em}
A \emph{flip book} of \emph{snapshots}.\\
\vspace{1em}
Allows for additional dynamics to be calculated between \emph{snaps}.\\(i.e frequency, valve position, \ldots )\\
\end{frame}
%------------------------------------------------
\subsection{Transient Simulation Differences}
%------------------------------------------------
\begin{frame}
\frametitle{Transient vs Long-Term Simulation}
Time scale.
Level of detail required.
Equations - transient simulation uses many ODEs to find next steady state and uses very small timesteps, 
PSLTDSim uses ODE to find next guess of certain inputs to the power flow which then computes the next steady state - uses much large time steps.
\end{frame}
%************************************************
\section{Recap}
%________________________________________________
\subsection{General Project Overview}
\begin{frame}
\frametitle{So, what's happening?}
Essentially:
\begin{itemize}
	\item Executing computer simulations of the western interconnection that are  over 10 minutes long.
	\item Simulation `time steps' are a sequence of power flows (\emph{snapshots})
	\item Additional dynamic calculations are performed between each `time step'.
\end{itemize}

\end{frame}
%------------------------------------------------
\begin{frame}
\frametitle{And why?}
To study engineering problems involving:
\begin{itemize}
	\item Long-term events (i.e. Wind Ramps)
	\item Multi-Area Power Interactions
	\begin{itemize}
		\item Inadvertent Interchange
		\item Turbine governor settings
		\item Automatic Generation Control Settings
		\item Governor and AGC interaction
	\end{itemize}
	\item Ways to reduce machine effort while meeting reliability standards.
\end{itemize}
\end{frame}
%************************************************
\section{Quick Results}
%------------------------------------------------
\begin{frame}
\frametitle{Quick Initial Validation}
pictures of step event comparisons of ltd vs psds
\end{frame}
%------------------------------------------------
\begin{frame}
\frametitle{Quick Controller Test}
BA controller action - do a with and without thing?
\end{frame}
%------------------------------------------------
%************************************************
\section{Final Bits}
%------------------------------------------------
\begin{frame}
\frametitle{Current Conclusions}
\begin{itemize}
	\item Software (PSLTDSim) output appears valid for small to medium size systems.
	\item Governor and AGC interactions can happen easily.
	\item Advanced control can be used to limit governor and AGC conflicts as well as reduce overall machine effort.
\end{itemize}
\end{frame}
%------------------------------------------------
\begin{frame}
\frametitle{Continuing Work}
\begin{itemize}
\item Experiments with AGC and turbine speed governor settings.
\item Use of valve travel and system reliability to gauge validity of control regime.
\item Expansion of software capabilities to handle full WECC.
\end{itemize}
\end{frame}
%------------------------------------------------

\begin{frame}
{\centering\Huge{Questions?}}
\end{frame}

%-------------------------------------------------
% attempt at having references in slide...
\begin{frame}[allowframebreaks]
\frametitle{References}
\renewcommand*{\bibfont}{\scriptsize} % makes references a resonable size
\printbibliography
\end{frame}



\begin{comment}
\frametitle{And why?}
\begin{itemize}
\item Simplification
\item Facilitate other research
\end{itemize}
\end{frame}
%------------------------------------------------
\begin{frame}
\frametitle{Engineering Areas of Interest}


%------------------------------------------------
\begin{frame}
\frametitle{Project Software Goals}
\begin{itemize}
\item Develop software for (long-term dynamic) LTD simulations.
\item Incorporate useful parts of GE software (PSLF):
\begin{itemize}
	\item Power Systems ($.sav$ files)
	\item Dynamic model data ($.dyd$ files)
	\item Power-flow solver
\end{itemize}
\item Create simplified dynamic models compatible with LTD time steps.
\end{itemize}
\end{frame}

%------------------------------------------------
%************************************************
\section{Background Information}
%________________________________________________

%************************************************
\section{Simulation Model}
%________________________________________________
\subsection{Assumptions, Coding Decisions, Approaches, and Software Operation.}
%------------------------------------------------
\begin{frame}
This simulation assumes:
\begin{enumerate}
\item Time steps of 0.5 to 1 second.
\item Fast dynamics are 'mostly' ignored.
\item System remains synchronized.
\item System frequency is described by the combined PU swing equation:
\setcounter{assumptions}{\value{enumi}} % allow for break in counting
\end{enumerate}
\[ \dot{\omega}_{sys} = \dfrac{1}{2H_{sys} } \left( \dfrac{P_{acc, sys} }{\omega_{sys}(t)} - D_{sys}\Delta\omega_{sys}(t)  \right)\] 
\begin{enumerate}
\setcounter{enumi}{\value{assumptions}}
\item No system damping $(D_{sys} = 0)$.
\end{enumerate}
\end{frame}
%------------------------------------------------

\begin{frame}
Software used:
\begin{itemize}
\item Python 3, IronPython
\item Erlang / AMQP
\item MATLAB
\end{itemize}
\end{frame}
%------------------------------------------------
\begin{frame}
\begin{itemize}
\item Agent\\ An autonomous individual object with properties and methods in a computer simulation.
\item Agent-Based Modeling\\ The idea that a system can be modeled using agents in an environment, and a description of agent-agent and agent-environment interactions. \tiny[2]
\end{itemize}
\end{frame}
%------------------------------------------------
\end{comment}
\end{document}