\documentclass[14pt, unknownkeysallowed]{beamer}

\usepackage{tikz}
\usepackage{comment}
\usepackage{fancyvrb}
\usepackage{tikz}
\usetikzlibrary{arrows,decorations.pathmorphing,backgrounds,positioning,fit,petri}
\usepackage{circuitikz} % for circuits!
\usetikzlibrary{arrows.meta} % for loads
\usepackage{float}
\usepackage{listings}

\usepackage{hyperref}

\usepackage[backend=biber,style=ieee, % bibliography style may be changed
sorting=nyt % sorts by author, year, title
]{biblatex}
\addbibresource{../Thesis/template/biliography.bib}
\nocite{*} % include non referenced citations
\setbeamertemplate{bibliography item}{\insertbiblabel} % number refs...?

\usepackage[default]{berasans}
\renewcommand*\familydefault{\sfdefault}  %% Only if the base font of the document is to be sans serif
\usepackage[T1]{fontenc}

\setbeamertemplate{navigation symbols}{}
\usetheme{Dresden}
\usecolortheme{seagull}
\usefonttheme{professionalfonts}


%\title{Long-Term Simulation of Power System Dynamics using Time Sequenced Power Flows}
\title{Long-Term Power System Dynamic Simulation using Time-Sequenced \\ Power Flows}
\author{Thad Haines}
\institute[MT TECH]{Montana Technological University - Master's Thesis Research Project}
\date{October 22nd, 2019}

\newcounter{assumptions}

\begin{document}
	
\begin{frame}
	\titlepage
\end{frame}

%************************************************
\section{Introduction}
%________________________________________________
\subsection{Explanation of Project}
%------------------------------------------------
\begin{frame}
\frametitle{What is Long-Term?}
% Insert timeScales Graph Here
Long-term describes the amount of time required for events of interest to occur and the simulation length to be executed. \\
\vspace{1em}
In this case: 10 to 60 minutes
\end{frame}
%------------------------------------------------
\begin{frame}
\frametitle{What is a Power System?}
% 3 Area MiniWECC picture?
Electrical supply connected to demand.\\
\vspace{1em}
Power plants connected via transmission lines to cities in different areas. \\
\vspace{1em}
A part of `The Grid'.
% picture of north american interconnections
\end{frame}
%------------------------------------------------
\begin{frame}
\frametitle{What is Dynamic Simulation?}
A computer's mathematical estimation of how a system will change over time.\\% in response to certain inputs and known starting conditions.\\ % Simulation of system changing over time
\vspace{1em}
How certain qualities of a power system may change over time in response to a known perturbance.
\end{frame}
%------------------------------------------------
\begin{frame}
\frametitle{What is a Power Flow?}
A steady state solution to all bus voltages, bus voltage angles, and real  and reactive power of a system.\\
\vspace{1em}
A \emph{snapshot} of a power system. \\% not dynamic
\vspace{1em}
Power flows are do not care about time.
\end{frame}
%------------------------------------------------
\begin{frame}
\frametitle{Time-Sequenced Power Flows?}
Multiple power flows arranged in a way to give the allusion of time.\\
\vspace{1em}
A \emph{flip book} of \emph{snapshots}.\\
\vspace{1em}
Allows for additional dynamics to be calculated between \emph{snaps}.\\(i.e frequency, valve position, \ldots )\\
\end{frame}
%------------------------------------------------
\begin{frame}
\frametitle{So, what's happening?}
Essentially:
\begin{itemize}
	\item Executing computer simulations of the power grid that are  over 10 minutes long.
	\item Simulation `time steps' are a sequence of power flows (\emph{snapshots})
	\item Additional dynamic calculations are performed between each `time step'.
\end{itemize}

\end{frame}
%------------------------------------------------
\begin{frame}
\frametitle{And why?}
To study engineering problems involving:
\begin{itemize}
	\item Long-term events (i.e. Wind Ramps)
	\item Multi-Area Power Interactions
	\begin{itemize}
		\item Inadvertent Interchange
		\item Turbine governor settings
		\item Automatic Generation Control Settings
		\item Governor and AGC interaction
	\end{itemize}
	\item Ways to reduce Machine effort while meeting reliability standards.
\end{itemize}
\end{frame}
%------------------------------------------------
\begin{frame}
\frametitle{Project Software Goals}
\begin{itemize}
\item Develop software for (long-term dynamic) LTD simulations.
\item Incorporate useful parts of GE software (PSLF):
\begin{itemize}
	\item Power Systems ($.sav$ files)
	\item Dynamic model data ($.dyd$ files)
	\item Power-flow solver
\end{itemize}
\item Create simplified dynamic models compatible with LTD time steps.
\end{itemize}
\end{frame}

%------------------------------------------------
%************************************************
\section{Background Information}
%________________________________________________
\subsection{Key Concepts Involved with Power Systems}
%------------------------------------------------
\begin{frame}
Frequency, Accelerating power and Inertia.\\
Direct link - electric demand always met. If there isn't enough generation, the kinetic energy stored as a moving inertia in a generator is converted to electric energy and the generator slows down.
\end{frame}
%------------------------------------------------
\begin{frame}
Governors Defined:\\
Turbine speed governors adjust a machines mechanical power to stop frequency decline. Input is frequency deviation and current operating set point. Primary control.
\end{frame}
%------------------------------------------------
\begin{frame}
Automatic Governor Control Defined\\
(AKA Load Frequency Control)
\\
Adjusts generator nominal operating set point to remove any inadvertent interchange and restore system frequency to 60 Hz. Secondary Control.
\end{frame}
%------------------------------------------------
\begin{frame}
Multi Area Interactions \\

Areas import or export power to each other.
\end{frame}
%************************************************
\section{Simulation Model}
%________________________________________________
\subsection{Assumptions, Coding Decisions, Approaches, and Software Operation.}
%------------------------------------------------
\begin{frame}
This simulation assumes:
\begin{enumerate}
\item Time steps of 0.5 to 1 second.
\item Fast dynamics are 'mostly' ignored.
\item System remains synchronized.
\item System frequency is described by the combined PU swing equation:
\setcounter{assumptions}{\value{enumi}} % allow for break in counting
\end{enumerate}
\[ \dot{\omega}_{sys} = \dfrac{1}{2H_{sys} } \left( \dfrac{P_{acc, sys} }{\omega_{sys}(t)} - D_{sys}\Delta\omega_{sys}(t)  \right)\] 
\begin{enumerate}
\setcounter{enumi}{\value{assumptions}}
\item No system damping $(D_{sys} = 0)$.
\end{enumerate}
\end{frame}
%------------------------------------------------

\begin{frame}
Software used:
\begin{itemize}
\item Python 3, IronPython
\item Erlang / AMQP
\item MATLAB
\end{itemize}
\end{frame}
%------------------------------------------------
\begin{frame}
\begin{itemize}
\item Agent\\ An autonomous individual object with properties and methods in a computer simulation.
\item Agent-Based Modeling\\ The idea that a system can be modeled using agents in an environment, and a description of agent-agent and agent-environment interactions. \tiny[2]
\end{itemize}
\end{frame}
%------------------------------------------------

%************************************************
\section{Final Bits}
%------------------------------------------------
\begin{frame}
\frametitle{Current Conclusions}
\begin{itemize}
	\item Software (PSLTDSim) produces valid output for small to medium size systems.
	\item Governor and AGC interactions can happen easily
	\item Deadbands and conditional logic should be used to limit governor and AGC conflicts
\end{itemize}
\end{frame}
%------------------------------------------------
\begin{frame}
\frametitle{Continuing Work}
\begin{itemize}
\item Further experiment with AGC and turbine speed governor settings
\item Use valve travel and system reliability as engineering measuring sticks.
\end{itemize}
\end{frame}
%------------------------------------------------

\begin{frame}
\centering{\Huge{Questions?}}
\end{frame}

%-------------------------------------------------
% attempt at having references in slide...
\begin{frame}[allowframebreaks]
\frametitle{References}
\renewcommand*{\bibfont}{\scriptsize} % makes references a resonable size
\printbibliography
\end{frame}



\begin{comment}
\frametitle{And why?}
\begin{itemize}
\item Simplification
\item Facilitate other research
\end{itemize}
\end{frame}
%------------------------------------------------
\begin{frame}
\frametitle{Engineering Areas of Interest}
\end{comment}
\end{document}